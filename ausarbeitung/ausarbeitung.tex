\input{einstellungen/grundeinstellungen} % Die stilistischen Parameter

\date{31.08.2012}
%---------------------------------------------------------------------------------------------------
% Anfang des Schriftstücks
%---------------------------------------------------------------------------------------------------
\begin{document}

%---------------------------------------------------------------------------------------------------
% Erstellen des Deck- und des Titelblatts
%---------------------------------------------------------------------------------------------------
\createCover{Erweiterung des Routing-Atlas\\
WiSe 2012/2013 - 28.02.2013} % Art der Arbeit
{Andreas Krohn} % Author
{Ausarbeitung: Seminar} % Title

%---------------------------------------------------------------------------------------------------
% Zusammenfassung}
%---------------------------------------------------------------------------------------------------
% %---------------------------------------------------------------------------------------------------
% Zusammenfassung
%---------------------------------------------------------------------------------------------------
\newpage
\thispagestyle{empty}
\subsection*{Martin Mustermann}
\abstractentry{Thema der Bachelorarbeit}{Mein Titel der Arbeit: Sollte kurz und interessant klingen und nicht die gesamte Aufgabenstellung und Abgrenzungen beinhalten.}
\abstractentry{Stichworte}{...die wichtigsten Stichwörter}
\abstractentry{Kurzzusammenfassung}{
In dieser Arbeit wurde....
}

\selectlanguage{english}
\subsection*{Martin Mustermann}
\abstractentry{Title of the paper}{My title of the paper: Should sound short and interesting and should not contain the complete type of problem and delimitations.}
\abstractentry{Keywords}{...the most important keywords}
\abstractentry{Abstract}{
In this paper...
}
\selectlanguage{ngerman}


%---------------------------------------------------------------------------------------------------
% Verzeichnisse
%---------------------------------------------------------------------------------------------------
\tableofcontents % Inhaltsverzeichnis
% \listoftables % Tabellenverzeichnis
% \listoffigures % Abbildungsverzeichnis

%---------------------------------------------------------------------------------------------------
% Einführung
%---------------------------------------------------------------------------------------------------
\newpage

\section{Einführung}

Ziel des Masterseminars ist die Entwicklung und Eingrenzung des Themas für die Masterarbeit.
Dabei sind Vorarbeiten der vergangenen Semester zu berücksichtigen.

\subsection{Motivation}
Das Internet ist nicht an nationale Grenzen gekoppelt und ursprünglich eher als Verbund autonomer Entitäten entworfen.
Da mittlerweile ein großer Teil der Kommunikation und des Geschäftslebens das Internet nutzt, sehen Staaten seit Jahren zunehmend den Bedarf regulierend einzugreifen.
Die Untersuchung der Topologie unter Berücksichtigung der Staaten wird hier interessant.
Da es keine allwissende übernationale Registrierungs- und Regulierungsentität gibt, lässt sich nur so abschätzen, welche dritten Jurisdiktionen an einer Kommunikation zweier Partner beteiligt sind.

Angesichts dieser Tatsache scheint es sinnvoll hierfür ein Werkzeug zu entwickeln.
Der Bau eines solches Werkzeug bietet verschiedene Herausforderungen, die einer Lösung bedürfen.
\begin{itemize}
  \item Zuordnung von Bestandteilen des Internets zu Ländern
  \item Sammeln und Archivieren von Daten um Entwicklungen über Zeit beobachten zu können
  \item Modellierung des policybasierten Routings im Internet
  \item Validierung dieses Modells
\end{itemize}
Hauptsächlich besteht das Problem darin, dass die Sicht auf das Internet abhängig vom Standort der Beobachtung ist und darüber hinaus nie „komplett“ ist.

% \subsection{Umfeld und Vorarbeiten}
% Die zu erstellende Masterarbeit ist im Bereich der Topologieanalyse des Internets verortet.

\section{Routing-Atlas}\label{sec:routingatlas}

%---------------------------------------------------------------------------------------------------
% Ausblick
%---------------------------------------------------------------------------------------------------
\section{Ausblick}\label{sec:ausblick}

%---------------------------------------------------------------------------------------------------
% Schluss
%---------------------------------------------------------------------------------------------------
\section{Zusammenfassung}\label{sec:schluss}


%---------------------------------------------------------------------------------------------------
% Literaturverzeichnis
%---------------------------------------------------------------------------------------------------
%\bibliographystyle{dinat} % Anpassung an deutsche Zitierweise: Alphabetische Sortierung, Abkürzungen
%\bibliographystyle{unsrt}
\bibliographystyle{IEEEtranN} % dinat.bst fehlt, kein format für @ONLINE, gefällt mir so irgendwie besser..
\bibliography{literatur/literatur2} % Literaturverzeichnis

%---------------------------------------------------------------------------------------------------
% Ende des Schriftstücks
%---------------------------------------------------------------------------------------------------
\end{document}
