%---------------------------------------------------------------------------------------------------
% Vorarbeiten
%---------------------------------------------------------------------------------------------------
\section{Vorarbeiten}\label{sec:previous}

In den vergangenen Semestern wurde auf verschiedene Aspekte des Routing-Atlas eingegangen.
Diese Vorarbeiten sind Inhalt der folgenden Abschnitte.

\subsection{Anwendung 1}
In der Veranstaltung Anwendung 1 wurde das Projekt Routing-Atlas in seiner damaligen Form vorgestellt~\cite{KrohnAW1}.

Der Routing-Atlas nutzt ein Modell der Routingtopologie auf AS-Ebene, das Rolf Winter an den NEC Labs Europe erstellt hat~\cite{neclab-topology, Winter:2009:MIR:1577959.1577976}.
Nach Abschluss des aus europäischen Mitteln finanzierten Rahmenprojekts „Trinity” wurde die regelmäßige Neuberechnung dieses Modells mit aktuellen Daten eingestellt.
Damit bestand für den Routing-Atlas der Bedarf auf anderem Weg ein aktuelles Modell der Topologie zu erhalten.
Das damalige Primärziel des Autors war der Ersatz dieser externen Datenquelle.
% Die Schaffung dieses Modells war zunächst Ziel des Autors.

Perspektivisch sollte eine weitesgehende Automatisierung der Toolchain des Routing-Atlas erreicht werden, damit regelmäßig Daten und Graphen veröffentlicht werden können.
Weiterhin schien die Anpassung und Anwendung der Toolchain für andere Länder und auf das IPv6-Protokoll ein lohnendes Ziel zu sein.

\subsection{Anwendung 2}
Vortrag und Ausarbeitung~\cite{KrohnAW2} im Rahmen der Veranstaltung Anwendung 2 stellte verschiedene andere Projekte und Arbeiten zur Topologieanalyse und -inferenz vor.



\subsection{Projekt 1}
Der erste Anlauf für Projekt 1 beschäftigte sich mit der Modellierung der Topologie des Internets auf AS-Ebene.
Dazu wurden die aus BGP-Updates und -Dumps extrahierten AS-Links der UCLA~\cite{ucla-topology, Zhang:2005:CIA:1052812.1052825} genutzt.
Auf diesen basierend sollten kürzesten Pfade für alle Paare der ca. 44500 in der Datenbasis vorhandenen ASe berechnet werden. %all-pair-shortest-path Algorithmus implementiert
Die resultierenden Pfade sollten möglichst das tatsächliche Routing im Internet nachbilden.

Die zwei Herausforderungen hierbei sind im Wesentlichen:
\begin{itemize}
  \item Die Komplexität der Berechnung - sowohl im Bezug auf Zeit als auch auf den Speicherbedarf - sowie die effiziente Implementierung.
  \item Die Modellierung des policybasierten Routings durch die Wahl passender Kantengewichte. Ralf Winter gibt in seinem Paper~\cite{Winter:2009:MIR:1577959.1577976} Hinweise, aber keine ausführlichen und (leicht) nachvollziehbare Details auf seine Wahl der Kantengewichte.
\end{itemize}

Es wurden verschiedene Algorithmen zur Lösung des all-pair-shortest-path Problems implementiert.
Dabei traten je nach Wahl der Datenstruktur für Knoten und Kanten des Graphen ab einer gewissen Größe der Datenmenge Probleme mit Laufzeit oder Speicherbedarf auf.

Letztendlich ist dieser Weg obsolet, da es bereits Tools gibt, die das Problem in annehmbarer Zeit lösen, wie zum Beispiel das Paket igraph~\cite{r-igraph} für R~\cite{r}.

Als kleines Subprojekt wurde die Länderzuordnung des Routing-Atlas mittels einer Geo-IP Datenbank validiert.
