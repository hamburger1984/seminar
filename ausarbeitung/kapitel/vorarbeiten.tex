%---------------------------------------------------------------------------------------------------
% Vorarbeiten
%---------------------------------------------------------------------------------------------------
\section{Vorarbeiten}\label{sec:previous}

In den vergangenen Semestern wurde auf verschiedene Aspekte des Routing-Atlas eingegangen.
Diese Vorarbeiten sind Inhalt der folgenden Abschnitte.

\subsection{Anwendung 1}
In der Veranstaltung Anwendung 1 wurde das Projekt Routing-Atlas in seiner damaligen Form vorgestellt~\cite{KrohnAW1}.

Der Routing-Atlas nutzt ein Modell der Routingtopologie auf AS-Ebene, das Rolf Winter an den NEC Labs Europe erstellt hat~\cite{neclab-topology, Winter:2009:MIR:1577959.1577976}.
Nach Abschluss des aus europäischen Mitteln finanzierten Rahmenprojekts „Trinity” wurde die regelmäßige Neuberechnung dieses Modells mit aktuellen Daten eingestellt.
Damit bestand für den Routing-Atlas der Bedarf auf anderem Weg ein aktuelles Modell der Topologie zu erhalten.
Das damalige Primärziel des Autors war der Ersatz dieser externen Datenquelle.
% Die Schaffung dieses Modells war zunächst Ziel des Autors.

Perspektivisch sollte eine weitesgehende Automatisierung der Toolchain des Routing-Atlas erreicht werden, damit regelmäßig Daten und Graphen veröffentlicht werden können.
Weiterhin schien die Anpassung und Anwendung der Toolchain für andere Länder und auf das IPv6-Protokoll ein lohnendes Ziel zu sein.

\subsection{Anwendung 2}
Vortrag und Ausarbeitung~\cite{KrohnAW2} im Rahmen der Veranstaltung Anwendung 2 stellte verschiedene andere Projekte und Arbeiten zur Topologieanalyse und -inferenz vor.


\subsection{Projekt 1}

