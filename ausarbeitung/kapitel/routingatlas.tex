%---------------------------------------------------------------------------------------------------
% Routing-Atlas
%---------------------------------------------------------------------------------------------------
\section{Routing-Atlas}\label{sec:routingatlas}

Das Internet ist nicht an nationale Grenzen gekoppelt und ursprünglich eher als Verband autonomer Entitäten entworfen.
Da mittlerweile ein großer Teil der Kommunikation und des Geschäftslebens das Internet nutzt, sehen Staaten seit Jahren zunehmend den Bedarf regulierend einzugreifen.
Die Untersuchung der Topologie mit Bezug auf Staaten wird hier interessant, weil sich nur so abschätzen lässt, welche dritten Jurisdiktionen an einer Kommunikation zweier Partner beteiligt sind.

Der Routing-Atlas ist ein Projekt der inet-AG in Zusammenarbeit mit dem BSI.
Ziel ist die Untersuchung der Topologie des Internets auf AS-Ebene.
Gegenüber vorherigen Ansätzen wurde hierbei ein besonderer Fokus auf die die Kriterien Staat und Branche gelegt.
Dabei sollte geklärt werden, in wie weit eine nationale Klassifizierung des Internets auf IP- und AS-Ebene möglich ist, in wie weit das Routing einer Nation in sich abgeschlossen ist und wie starke Abhängigkeiten bestehen.

Das Routing-Atlas Projekt nutzt verschiedene Datenquellen und führt diese zusammen.
\begin{itemize}
  \item Informationen zu IP-Adressblöcken und -Präfixen sowie zu ASen aus der RIPE-Datenbank~\cite{ripe-db} und vom Team Cymru~\cite{cymru}
  \item Datensätze regionaler Monitore~\cite{ripe-ris}
  \item Matrizen zu Shortest Path und Next Hop des NEC labs, die Rolf Winter erstellt hat~\cite{neclab-topology, Winter:2009:MIR:1577959.1577976}
  \item Hierarchische Einordnung der ASe auf Basis von Daten, die von einer Arbeitsgruppe der UCLA um Lixia Zhang gesammelt werden~\cite{ucla-topology, Zhang:2005:CIA:1052812.1052825}
\end{itemize}
