%---------------------------------------------------------------------------------------------------
% Ausblick
%---------------------------------------------------------------------------------------------------
\section{Ausblick}\label{sec:ausblick}
Die nächste Aufgabe ist ein neuer Anlauf für das Projekt 1.
Der Fokus liegt hier auf der Untersuchung aktiver Messverfahren für die Gewinnung weiterer Informationen zu AS-Links.

\subsection{Vorgehensweise}
Zunächst soll das aktive Messen á la Augustin~\cite{Augustin:2009:IM:1644893.1644934} nachempfunden werden.
Über Kontakte besteht die Möglichkeit an einem deutschen IXP Messungen durchzuführen.
Weiterhin sind Ausschnitte der Topologie im Umfeld dieses IXP bekannt (oder sie können erfragt werden).
In einem ersten Schritt sollen diese Ausschnitte mittels traceroute und anderer Verfahren vermessen werden.
Dabei sind unter anderem die folgenden Fragen zu beantworten:
\begin{itemize}
  \item Antworten die gesuchten Bestandteile der Infrastruktur auf \texttt{ICMP ECHO\_REQUESTS}?
  \item Antworten sie „wahrheitsgemäß“?
  \item Lassen sich einzelne Lücken im traceroute auf anderen Wegen ermitteln?
  \item Liefert traceroute im Vergleich zur bisherigen auf passiven Messungen basierenden Topologie weitere Informationen?
\end{itemize}

Das Ziel ist ein Verfahren zu entwickeln, dass zuverlässig korrekte zusätzliche Informationen liefert.
Diese sollen in eine passende Form gebracht werden und der Datenbasis des Routing-Atlas zugefügt werden.
